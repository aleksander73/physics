\documentclass{article}
\usepackage[utf8]{inputenc}
\usepackage{amsmath}
\usepackage{amssymb}

\title{1000 solved problems in modern physics}
\author{Aleksander Dobkowski}
\date{}

\begin{document}
\maketitle

\section{Mathematical physics}
\subsection{Vector calculus}

% ===================================================

\paragraph{Ex 1.1}
If $\phi=\frac{1}{r}$, where $r = \left( x^2 + y^2 + z^2 \right)^{\frac{1}{2}}$, show that $\nabla \phi = \frac{r}{r^3}$
\paragraph{Solution}

\[ \phi = \frac{1}{\left( x^2 + y^2 + z^2 \right)^{\frac{1}{2}}} = 
\left( x^2 + y^2 + z^2 \right)^{-\frac{1}{2}} \]

Where
\[ x y z \not = 0 \]

\[ \nabla \phi =
\left( \frac{\partial}{\partial x} \hat{i} + \frac{\partial}{\partial y} \hat{j} + \frac{\partial}{\partial z} \hat{k} \right) \left( x^2 + y^2 + z^2 \right)^{-\frac{1}{2}}\]

Let's calculate the derivative with respect to $x$ using the chain rule ...
\[ \frac{\partial \phi}{\partial x} = 
\frac{\partial \phi}{\partial u} \frac{\partial u}{\partial x} \] 
\[ u = x^2 + y^2 + z^2 \]

\[ \frac{\partial \phi}{\partial x} = 
\frac{\partial}{\partial u} u^{-\frac{1}{2}} \frac{\partial}{\partial x} \left( x^2 + y^2 + z^2 \right) = \frac{-u^{-\frac{3}{2}}}{2} 2x = 
-u^{-\frac{3}{2}}x = 
-\left( x^2 + y^2 + z^2 \right)^{-\frac{3}{2}}x \]

The derivatives with respect to y and z look like
\[ \frac{\partial \phi}{\partial y} = -\left( x^2 + y^2 + z^2 \right)^{-\frac{3}{2}}y \]
\[ \frac{\partial \phi}{\partial z} = -\left( x^2 + y^2 + z^2 \right)^{-\frac{3}{2}}z \]

Finally
\[ \nabla \phi = 
-\left( x^2 + y^2 + z^2 \right)^{-\frac{3}{2}} x\hat{i} 
-\left( x^2 + y^2 + z^2 \right)^{-\frac{3}{2}} y\hat{j}
-\left( x^2 + y^2 + z^2 \right)^{-\frac{3}{2}} z\hat{k}\]
\[ = -\left( x^2 + y^2 + z^2 \right)^{-\frac{3}{2}} \left( x\hat{i} + y\hat{j} + z\hat{k} \right) \]

% ===================================================

\section{Quantum mechanics I}
\subsection{de Broglie waves}

% ===================================================

\paragraph{Ex 2.1}
(a) Write down the equation relating the energy \textit{E} of a photon to its frequency
\textit{$\nu$}. Hence determine the equation relating the energy \textit{E} of a photon to its
wavelength. (b) A $\pi^0$ meson at rest decays into two photons of equal energy. What is the
wavelength (in m) of the photons? (The mass of the $\pi^0$ is 135 $M e V/c^2$)
\paragraph{Solution}

\paragraph{(a)}
\begin{itemize}
    \item \text{Equation relating energy of a photon to its frequency $\nu$}
        \[ E = h \nu \]
    \item \text{Equation relating energy of a photon to its wavelength $\lambda$}
        \[ c = \frac{\lambda}{T} = \lambda \nu \]
        \[ \nu = \frac{c}{\lambda} \]
        \[ E = h \nu  = \frac{h c}{\lambda} \]
\end{itemize}

\paragraph{(b)}
\paragraph{Data}
\begin{itemize}
    \item $m_\pi$ = 135 $M e V/c^2$ = $2.4165 \cdot 10^{-28}$ kg
\end{itemize}

A decaying meson's energy \textit{$E_\pi$} will be
\[ E_\pi = m_\pi c^2 \]

The meson decays into 2 photons of equal energy \textit{$E_\gamma$}
\[ E_\gamma = \frac{E_\pi}{2}\]

Since
\[ E_\gamma = h \nu  = \frac{h c}{\lambda} \]
\[ \lambda = 
\frac{h c}{E_\gamma} = 
\frac{2 h c}{E_\pi} = 
\frac{2 h c}{m_\pi c^2} = 
\frac{2 h}{m_\pi c} \]

Finally
\[ \lambda = \frac{2 \cdot 6.626 \cdot 10^{-34} J s}{2.4165 \cdot 10^{-28} kg \cdot 299792458 \frac{m}{s}} = 1.83 \cdot 10^{-14} m \]

% ===================================================

\paragraph{Ex 2.2}
Calculate the wavelength in \textit{nm} of an electron which has been accelerated from rest through a potential difference of 54 V.
\paragraph{Solution}

\paragraph{Data}
\begin{itemize}
    \item $V = 54$ V 
\end{itemize}

Having passed through the potential $V$, each electron will gain the energy of 54 electron-volts, which expressed in Joules gives the energy

\[\Delta E = e V\]

Since the electron has been accelerated from the state of rest, the increase in its energy will be its total energy E.

\[E = \Delta E\]

From the equation

\[\lambda = \frac{h}{p}\]

... the wavelength can be calculated having found the momentum of the electron.\\

The work done accelerating the electron will be equal to the electron's kinetic energy after it passes through the potential difference.

\[e V = \frac{mv^2}{2}\]

... and multiplying the right side by $\frac{m}{m}$

\[e V = \frac{m^2v^2}{2m}\]
\[e V = \frac{p^2}{2m}\]
\[p = \sqrt{2 m e V}\]

Now, the wavelength of the electron can be described as a function $\lambda$ of the potential through which it passed

\[\lambda (V) = \frac{h}{\sqrt{2 m e V}}\]

Finally

\[\lambda (54V) = \frac{6.626 \cdot 10^{-34} Js}{\sqrt{2 \cdot 9.109 \cdot 10^{-31} kg \cdot 1.602 \cdot 10^{-19} C \cdot 54V}} = 0.1669 nm\]

% ===================================================

\paragraph{Ex 2.3}
Show that de Broglie wavelength for neutrons is given by $\lambda = 0.286 \AA / \sqrt{E}$, where E is in electron-volts.
\paragraph{Solution}

From exercise 2.2 we know that

\[\lambda = \frac{h}{\sqrt{2 m E_{J}}}\]

Accounting for the fact that the energy E is expressed in electron-volts and the equation above expects energy expressed in Joules, we convert the energy E  from eV to Joules.

\[\lambda = \frac{h}{\sqrt{2 m E_{J}}} =  \frac{h}{\sqrt{3.204 \cdot 10^{-19} \cdot m E}}\]

Having gathered all the constants in one constant k.

\[\lambda = \frac{h}{\sqrt{3.204 \cdot 10^{-19} \cdot m}} \cdot \frac{1}{\sqrt{E}} = \frac{k}{\sqrt{E}}\]

\[k = 2.86 \cdot 10^{-11} = 0.286 \AA\]

Finally

\[\lambda = \frac{0.286 \AA}{\sqrt{E}}\]

% ===================================================

\paragraph{Ex 2.4}
Show that if an electron is accelerated through V volts then de Broglie wavelength in angstroms is given by $\lambda = \sqrt{\frac{150}{V}}$
\paragraph{Solution}

From exercise 2.2 we know that

\[\lambda = \frac{h}{\sqrt{2 m e V}}\]

Having gathered all the constants in one constant k.

\[\lambda = \frac{h}{\sqrt{2 m e}} \cdot \frac{1}{\sqrt{V}} = \frac{k}{\sqrt{V}}\]

\[k = 1.236 \cdot 10^{-9} = 12.36 \AA \approx \sqrt{150} \AA\]

Finally

\[\lambda \approx \sqrt{\frac{150}{V}} \AA\]

% ===================================================

\paragraph{Ex 2.5}
(1) A thermal neutron has a speed $v$ at temperature $T = 300K$ and kinetic energy $\frac{m_{n}v^2}{2} = \frac{3kT}{2}$. (a) Calculate its de Broglie wavelength. (b) State whether a beam of these neutrons could be diffracted by a crystal, and why?\\(2) Use Heisenberg's Uncertainty principle to estimate the kinetic energy (in $MeV$) of a nucleon bound within a nucleus of radius $10^{-15}$ m.
\paragraph{Solution}

\paragraph{1}

\paragraph{(a)}

\[\frac{m_n v^2}{2} = \frac{3kT}{2}\]

\[v = \sqrt{\frac{3kT}{m_n}}\]

De Broglie wavelength is described as

\[\lambda = \frac{h}{p} = \frac{h}{m_{n}v}\]

... we substitute $v$ and get

\[\lambda = \frac{h}{\sqrt{3km_{n}T}}\]

Now the wavelength can be described as a function $\lambda$ of the temperature.

\[\lambda(T) = \frac{h}{\sqrt{3km_{n}T}}\]

Finally

\[\lambda(300K) = \frac{6.626 \cdot 10^{-34} Js}{\sqrt{3 \cdot 1.381 \cdot 10^{-23} m^{2}kgs^{-2}K^{-1} \cdot 1.675 \cdot 10^{-27} kg \cdot 300K}} = 1.452 \AA\]

\paragraph{(b)}

Given that the de Broglie wavelength $\lambda$ is similar to the distance between atoms in crystal $d$.

\[\lambda \approx d\]

\paragraph{2}

Heisenberg's Uncertainty principle states that

\[\Delta x \Delta p \gtrsim \hbar\]

\[ \Delta p \gtrsim \frac{\hbar}{\Delta x} \]

We assume that we make the most precise measurement possible. Therefore the inequality becomes an equation and we skip the $\Delta$ notation.

\[ p = \frac{\hbar}{x} \]

The total energy of the particle is formulated as follows

\[ E^2 = p^2c^2 + m^2c^4 \Rightarrow E = \sqrt{p^2c^2 + m^2c^4} \]

We also know that  the total energy of a particle is the sum of its rest energy and kinetic energy.

\[ E = mc^2 + E_k \] 
\[ E_k = E - mc^2 = \sqrt{p^2c^2 + m^2c^4} - mc^2 = \sqrt{\left( \frac{\hbar c}{x} \right)^2 + m^2c^4} - mc^2 \]

Now the kinetic energy of the particle can be expressed as a function $E_k$ of the particle's mass $m$ and the nucleus radius $x$.

\[ E_k(m, x) = \sqrt{\left( \frac{\hbar c}{x} \right)^2 + m^2c^4} - mc^2 \]

% ===================================================

\paragraph{Ex 2.6}
The relation for total energy ($E$) and momentum ($p$) for a relativistic particle is $E^2 = p^2c^2 + m^2c^4$, where $m$ is the rest mass and $c$ is the velocity of light. Using the relativistic relations $E = \hbar \omega$ and $p = \hbar k$, where $\omega$ is the angular frequency and $k$ is the wave number, show that the product of group velocity ($v_g$) and the phase velocity ($v_p$) is equal to $c^2$, that is $v_g v_p = c^2$.
\paragraph{Solution}

We substitute $E$ and $p$ in the equation and get
\[ \hbar^2 \omega^2 = \hbar^2 k^2 c^2 + m^2 c^4 \]

We calculate $\omega$ from the above equation
\[ \omega = \left( k^2 c^2 + m^2 c^4 \hbar^{-2}  \right)^{\frac{1}{2}} \] 

We now can calculate the group and phase velocities

1. Group velocity
\[ v_g = \frac{\partial \omega}{\partial k} = \frac{\partial}{\partial k} \left( k^2 c^2 + m^2 c^4 \hbar^{-2} \right)^{\frac{1}{2}} = \]
\[ = \frac{1}{2} \left( k^2 c^2 + m^2 c^4 \hbar^{-2} \right)^{-\frac{1}{2}} \, \frac{\partial}{\partial k} \left( k^2 c^2 + m^2 c^4 \hbar^{-2} \right) = \]
\[ = \frac{1}{2} \left( k^2 c^2 + m^2 c^4 \hbar^{-2} \right)^{-\frac{1}{2}} \cdot 2kc^2 = kc^2 \left( k^2 c^2 + m^2 c^4 \hbar^{-2} \right)^{-\frac{1}{2}} \]

2. Phase velocity
\[ v_p = \frac{\omega}{k} = \frac{\left( k^2 c^2 + m^2 c^4 \hbar^{-2} \right)^{\frac{1}{2}}}{k} \]

We can now calculate the value of the expression $v_g v_p$

\[ v_g v_p = kc^2 \left( k^2 c^2 + m^2 c^4 \hbar^{-2} \right)^{-\frac{1}{2}} \cdot \frac{\left( k^2 c^2 + m^2 c^4 \hbar^{-2} \right)^{\frac{1}{2}}}{k} = c^2 \]

% ===================================================


% ===================================================

\end{document}
